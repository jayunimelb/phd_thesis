\bfseries{Abstract}\mdseries\\                                                                                                
                                                                                                                             

Galaxy clusters are powerful probes of cosmology. 
Their abundance depends on the rate of structure growth and the expansion rate of the universe, making cluster density highly sensitive to dark energy. 
%Galaxy clusters are one of the most sensitive probes of dark energy because their abundance depends on the rate of structure growth and the expansion rate of the universe. 
In addition to that galaxy clusters provide powerful constraints on matter density, matter fluctuation amplitude, and sum of neutrino masses.
 However, they are currently limited by systematic uncertainties in their mass estimation.
 Gravitational lensing is one of the most promising techniques to measure cluster mass as it directly probes the total matter content of the cluster. 
 The gravitational lensing source can either be optical galaxies or the cosmic microwave background (CMB). %and the latter being the main focus of this thesis.
 
 This thesis focuses on developing statistical and mathematical tools to robustly extract the cluster lensing signal from CMB data. 
 We developed a maximum likelihood estimator to optimally extract cluster lensing signal from the CMB polarisation and temperature data. 
 We found that the Stokes QU maps and the traditional EB maps extract the same amount of information. 
  We quantified the effect of foregrounds on CMB cluster lensing analysis. %and used realistic simulations to forecast the mass uncertainties for future surveys. 
  While the foregrounds set an effective noise floor for temperature estimator, the polarisation estimator is largely unaffected. 
  We used realistic simulations to forecast for future surveys, CMB cluster lensing is expected to constrain cluster mass at 3-6\%(1\%) level for stage-3 (stage-4) CMB experiments. 
  
  The thermal Sunyaev-Zel'dovich effect (tSZ) acts as a major contaminant in CMB cluster lensing analysis and induces significant systematic and statistical uncertainty.
   We developed modified quadratic estimator to eliminate the tSZ bias and template fitting approach to significantly reduce the tSZ statistical uncertainty.
  Using our modified quadratic estimator we constrained the mass of Dark Energy Survey year-3 cluster catalog. In addition to that we also constrained the normalisation parameter of optical richness-mass scaling relation. 
  The template approach is expected to reduce tSZ uncertainty by 40\% for future low noise CMB-surveys.  
  
 
  
                                                                                                                                   
\vspace{4.0cm}                                                                                                                                  
                                                                                                                                                
                                                                                                                                                
%\scriptsize{...............................................................................................................}\normalsize\\      

%                                        Joe Namesson\\
