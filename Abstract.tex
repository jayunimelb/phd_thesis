\bfseries{Abstract}\mdseries\\                                                                                                
                                                                                                                             

Galaxy clusters are powerful probes of cosmology. 
Their abundance depends on the rate of structure growth and the expansion rate of the universe, making the density of clusters highly sensitive to dark energy. 
%Galaxy clusters are one of the most sensitive probes of dark energy because their abundance depends on the rate of structure growth and the expansion rate of the universe. 
Galaxy clusters additionally provide powerful constraints on matter density, matter fluctuation amplitude, and sum of neutrino masses.
 However, cluster cosmology is currently limited by systematic uncertainties in the cluster mass estimation.
% The standard way to measure cluster mass is by using observable-mass scaling relation.
 Generally, the cluster masses are estimated using observable-mass scaling relations where the observable can be optical richness, X-ray temperature etc.
 The observable-mass scaling relation depends on the complex cluster baryonic physics which is not well understood and any deviation in the baryonic physics will lead to uncertainties in the mass estimation.
 On the other hand, gravitational lensing offers one of the most promising techniques to measure cluster mass as it directly probes the total matter content of the cluster. Gravitational lensing can additionally be used to calibrate the observable-mass scaling relations.
 The gravitational lensing source can either be optical galaxies or the cosmic microwave background (CMB). %and the latter being the main focus of this thesis.
 
My thesis focuses on developing statistical and mathematical tools to robustly extract the cluster lensing signal from CMB data. 
 We develop a maximum likelihood estimator to optimally extract cluster lensing signal from CMB data. 
 We find that the Stokes QU maps and the traditional EB maps provide similar constraints on mass estimates. 
  We quantify the effect of astrophysical foregrounds on CMB cluster lensing analysis. %and used realistic simulations to forecast the mass uncertainties for future surveys. 
  While the foregrounds set an effective noise floor for temperature estimator, the polarisation estimator is largely unaffected. 
  We use realistic simulations to forecast that CMB cluster lensing is expected to constrain cluster masses to 3-6\%(1\%) level for upcoming (next generation) CMB experiments. 
 
  
  One of the standard ways to extract the CMB-cluster lensing signal is by using the quadratic estimator. 
  The thermal Sunyaev-Zel'dovich effect (tSZ) acts as a major contaminant in quadratic estimator and induces significant systematic and statistical uncertainty.
   We develop modified quadratic estimator to eliminate the tSZ bias and to significantly reduce the tSZ statistical uncertainty.
  Using our modified quadratic estimator we constrain the mass of Dark Energy Survey year-3 cluster catalog. 
  We also put constraints on the normalisation parameter of optical richness-mass scaling relation. 
  In addition to removing the tSZ bias, modified quadratic estimator also reduces tSZ induced statistical uncertainty by 40\% in future low noise CMB-surveys.  
  
 
  
                                                                                                                                   
\vspace{4.0cm}                                                                                                                                  
                                                                                                                                                
                                                                                                                                                
%\scriptsize{...............................................................................................................}\normalsize\\      

%                                        Joe Namesson\\
