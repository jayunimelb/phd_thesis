\documentclass[usenatbib, twocolumn, nofootinbib, reprint, emulateapj, amsart]{revtex4-1}
\usepackage{newtxtext,newtxmath}
\usepackage[normalem]{ulem}
\usepackage{multirow}
\usepackage{booktabs}
\usepackage{color}
\usepackage{amsmath,amsfonts,bm}%,amsfonts,amsthm,bm}
\usepackage{hyperref}% add hypertext capabilities
% Depending on your LaTeX fonts installation, you might get better results with one of these:
%\usepackage{mathptmx}
%\usepackage{txfonts}
\usepackage{natbib	}
\usepackage[T1]{fontenc}
\usepackage{ae,aecompl}
\usepackage{relsize}

\usepackage{graphicx}	% Including figure files
\usepackage{amsmath}	% Advanced maths commands
\usepackage{amssymb}% Extra maths symbols

%%%%%%%%%%%%%%%%%%%%%%%%%%%%%%%%%%%%%%%%%%%%%%%%%%%%%%%%%%%%
%our macros
\newcommand{\sr}[1]{\textcolor{red}{[SR:] #1}}
\newcommand{\pending}[1]{\textcolor{red}{#1}}
\newcommand{\tbd}[1]{\textcolor{red}{#1}}
\newcommand{\refresponse}[1]{\textcolor{blue}{#1}}
\newcommand{\fittingradius}{10^{\prime}}
\newcommand{\arcmin}{^{\prime}}
\newcommand{\micron}{$\mu$}
\newcommand{\msolar}{\ensuremath{\mbox{M}_{\odot}}}
\newcommand{\nver}{\hat{\mathbf{n}}}
\newcommand{\snr}{$S/N$}
\newcommand{\bL}{\bm{\ell}}
\newcommand{\bnhat}{\bm{\hat{\mbox{n}}}}
\newcommand{\ukarcmin}{\ensuremath{\mu}{\rm K-arcmin}}
\newcommand{\cmmnt}[1]{}
%%experiments
\newcommand{\spt}{SPT-3G}
\newcommand{\mvir}{M_{200m}}
\newcommand{\msol}{M_{\odot}}
\newcommand{\munits}{\times 10^{14}\ \msol}

\newcommand{\advact}{AdvACT}
\newcommand{\so}{SO}
%our macros
%%%%%%%%%%%%%%%%%%%%%%%%%%%%%%%%%%%%%%%%%%%%%%%%%%%%%%%%%%%%
\begin{document}
\section{Cosmic Microwave Background cluster lensing}
Cosmic Microwave Background (CMB) photons while passing through intervening galaxy cluster gets lensed and we call this phenomenon as CMB-cluster lensing.
 Lensing remaps the unlensed CMB field based on the angular deflection caused by cluster gravitational potential.
 In mathematical form CMB cluster lensing can be written as the equation below
 \begin{eqnarray}
T(\hat{\textbf{n}})& = & \widetilde{T} (\hat{\textbf{n}} + \vec{\alpha}(\hat{\textbf{n}}))\\
Q(\hat{\textbf{n}}) & = & \widetilde{Q} (\hat{\textbf{n}} + \vec{\alpha}(\hat{\textbf{n}}))\\
U(\hat{\textbf{n}}) & = & \widetilde{U} (\hat{\textbf{n}} + \vec{\alpha}(\hat{\textbf{n}}))
\end{eqnarray}
where $ \widetilde{T}$ is unlensed temperature field, $\widetilde{U}$ and $\widetilde{Q}$ are the unlensed polarisation fields.
$\vec{\alpha}(\hat{\textbf{n}})$ denotes the deflection angle and is directly proportional to the mass of the galaxy cluster.

Lensing signal by an individual cluster is too weak to detect, so we need to stack many clusters to obtain a significant signal.
 For example, the lensing induced distortion due to a $2 \times 10^{14} \ M_{\odot}$ mass galaxy cluster  is $\sim 5.0$ and $0.5 \ \mu K$ in temperature and polarization respectively.
In this chapter, we will discuss various methods available in literature to extract this weak lensing signal.
\section{Quadratic Estimator}
%Here we give formalism of the temperature quadratic estimator; this can be extended to polarisation. 
%Equation 1 can be expanded in taylor series as follows 
%\begin{equation}
%T(\hat{\textbf{n}}) = \widetilde{T}(\hat{\textbf{n}}) + \nabla \widetilde{T} . \vec{\alpha} + ..
%\end{equation}
Typical size of galaxy cluster is of the order of few arc minutes. 
Primordial CMB doesn't have power at such small scales due to diffusion damping \cite{Silk} and can be approximated as a gradient. 
Lensing due to galaxy cluster induces a dipole kind of structure oriented along the direction of background gradient with hot and cold spots swapped.
For a given cluster mass and redshift, the magnitude of this dipole scales linearly with the magnitude of the CMB gradient. 
%Quadratic estimator is designed to exploit this correlation between large scale CMB gradient and small scale dipole.
This correlation between large scale CMB gradient and small scale dipole is known as the gradient approximation.
Below we provide the mathematical formalism for temperature quadratic estimator.%, as polarisation estimator follows.

 Under the gradient approximation, we construct an estimator of lensing convergence by multiplying the lensing map and the gradient map.
The gradient approximation doesn't hold for all fourier modes, only for the modes which are correlated by reconstruction.
Hence the maps are pre filtered in the fourier space to get minimum variance estimator
% As the gradient approximation induces correlation between ulensed temperature gradient and the lensed field, we form quadratic estimator by multiplying filtered temperature gradient and lensed temperature fields.  
% The lensed field $T_{L}(\hat{n})$ can be prefiltered in Fourier space to isolate modes for which gradient approximation is valid
 \begin{equation}
 \hat{k_{l}} = -A_{l} \int d^{2} \hat{n} e^{i\hat{n}.l} Re{\nabla .[G(\hat{n}) L^{*}(\hat{n})]}
 \end{equation}
 where $G(\hat{n})$, $L(\hat{n})$ are filtered gradient and lensing maps respectively, $A_{l}$ is the normalisation parameter.
  We obtain $G(\hat{n})$, $L(\hat{n})$ from the observed temperature map as follows
  \begin{eqnarray}
  L(\hat{n}) = \int \frac{d^{2}l}{(2\pi)^{2}} e^{il .\hat{n}} W^{T}_{l} T_{l}\\
  G(\hat{n}) = \nabla (\int\frac{d^{2}l}{(2\pi)^{2}} e^{il .\hat{n}} W^{TT}_{l} T_{l}   )
  \end{eqnarray}
 where $T_{l}$ is the observed temperature map in fourier space.  
 Fourier filters $W^{T}_{l}$ and $W^{TT}_{l}$ are given by 
 \begin{eqnarray}
 W^{T}_{l} = (C^{TT}_{l} + N^{TT}_{l})^{-1}\\
 W^{TT}_{l} =  \widetilde{C}^{TT}_{l}(C^{TT}_{l} + N^{TT}_{l})^{-1}
 \end{eqnarray}
 where  $\widetilde{C}^{TT}_{l}$,$C^{TT}_{l} $  is the unlensed and large scale structure lensed CMB power spectrum, $ N^{TT}_{l}$ is the noise power spectrum.
 
 \pending{plot showing the schematic representation of QE}
 
 \subsection{mitigating magnification bias}
Galaxy cluster magnifies the background image and decreases the observed temperature gradient behind it , which leads to a low bias in reconstruction.
The bias is due to the overlap in scales between the unlensed temperature gradient and the lensed temperature field. 
Though wiener filter reduces the bias, it is not removed completely.
We can reduce the bias further by exploiting the prior knowledge on unlensed CMB power spectrum.
 From fig 1 which shows the unlensed rms gradient as a function of multipoles, it is evident that most of the power for the gradient map comes from scales below l<2000.
 So by low pass filtering the gradient map, we separate the unlensed temperature gradient and the lensed temperature field with almost no loss in SNR.
 \pending{include wayne hu's plot}
 
 \subsection{fitting model}
To obtain the mass of the cluster, we need compare the observed lensing profile to the convergence model generated using an assumed halo mass profile.
Essentially we assume cluster to follow NFW profile and generate the model as function of mass $k(M)$. 
With model prediction in hand we can write down the likelihood of observing the data as:
\begin{equation}
-2 ln L (k | M) = (k - k_{m})(M) C^{-1} [(k - k_{m})]^{T}
\end{equation}
where k is the observed lensing convergence profile, $k_{m}$ is lensing convergence for NFW profile at mass M, C is the covariance matrix. 
 
\section{modified Quadratic Estiatmator}
While designed to pull the lensing induced correlations, QE is equally sensitive to any other signal present in both G and L maps.
The major systematic for lensing analysis is thermal Sunyaey -Zel'dovich (tSZ) effect, which is an order of magnitude greater than lensing signal biases the analysis if not taken into account. 
 Previous studies either used a tSZ free map or a more stringent low pass filtering of the gradient map.
 Both the method are suboptimal as they decrease the SNR. 
 
 During my thesis we came up with a new approach to completely remove the tSZ bias. 
 Any foreground signal which is present in both the maps will lead to a systematic bias, so by getting rid of the foreground signal in one of the maps (either G or L) should eliminate the bias completely.
  As shown in fig 1 most of the gradient power comes from multipoles l <2000 and CMB is not limited by noise at those scales.
  So, natural choice would be to eliminate tSZ in the gradient map by using linear combination of different frequencies. 
  

  


\section{Refined m QE}
\section{Maximum Likelihood Estimator}


\end{document}