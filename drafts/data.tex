%%%%%%%%%%%%%%%%%%%%%%%%%%%%%%%%%%%%%%%%%%%%%%%%%%
\documentclass[usenatbib, twocolumn, nofootinbib, reprint, emulateapj, amsart]{revtex4-1}
\usepackage{newtxtext,newtxmath}
\usepackage[normalem]{ulem}
\usepackage{multirow}
\usepackage{booktabs}
\usepackage{color}
\usepackage{amsmath,amsfonts,bm}%,amsfonts,amsthm,bm}
\usepackage{hyperref}% add hypertext capabilities
% Depending on your LaTeX fonts installation, you might get better results with one of these:
%\usepackage{mathptmx}
%\usepackage{txfonts}
\usepackage{natbib	}
\usepackage[T1]{fontenc}
\usepackage{ae,aecompl}
\usepackage{relsize}

\usepackage{graphicx}	% Including figure files
\usepackage{amsmath}	% Advanced maths commands
\usepackage{amssymb}% Extra maths symbols

%%%%%%%%%%%%%%%%%%%%%%%%%%%%%%%%%%%%%%%%%%%%%%%%%%%%%%%%%%%%
%our macros
\newcommand{\sr}[1]{\textcolor{red}{[SR:] #1}}
\newcommand{\pending}[1]{\textcolor{red}{#1}}
\newcommand{\tbd}[1]{\textcolor{red}{#1}}
\newcommand{\refresponse}[1]{\textcolor{blue}{#1}}
\newcommand{\fittingradius}{10^{\prime}}
\newcommand{\arcmin}{^{\prime}}
\newcommand{\micron}{$\mu$}
\newcommand{\msolar}{\ensuremath{\mbox{M}_{\odot}}}
\newcommand{\nver}{\hat{\mathbf{n}}}
\newcommand{\snr}{$S/N$}
\newcommand{\bL}{\bm{\ell}}
\newcommand{\bnhat}{\bm{\hat{\mbox{n}}}}
\newcommand{\ukarcmin}{\ensuremath{\mu}{\rm K-arcmin}}
\newcommand{\cmmnt}[1]{}
%%experiments
\newcommand{\spt}{SPT-3G}
\newcommand{\mvir}{M_{200m}}
\newcommand{\msol}{M_{\odot}}
\newcommand{\munits}{\times 10^{14}\ \msol}

\newcommand{\advact}{AdvACT}
\newcommand{\so}{SO}
%our macros
%%%%%%%%%%%%%%%%%%%%%%%%%%%%%%%%%%%%%%%%%%%%%%%%%%%%%%%%%%%%
\begin{document}
\section{South Pole Telescope}
The South Pole Telescope (SPT) is a 10 m diameter, wide-field, offset Gregorian telescope, located at the Amundsen-Scott South Pole station in Antartica. 
Atmosphere at south pole is extremely dry and exceptionally stable making it one of the best sites on earth for observations at millimeter and sub-millimeter wavelengths.
SPT has completed two surveys so far, SPT-SZ and SPTpol. 
SPT-SZ  covered 2500 $deg^{2}$ of southern sky at 90, 150, and 220 GHz; SPTpol is described in detail in the next section.


\section{SPT\lowercase{pol}}
In early 2012, a polarization sensitive camera was installed on SPT.
SPTpol observed 500 $deg^{2}$ of southern sky in both temperature and polarization at 90 and 150GHz. 


To complement the polarization-sensitive receiver, we modified the telescope to reduce ground pickup by installing a 1 m guard ring around the 10 m primary and a small ``snout" near prime focus at the top of the receiver cabin in 2012.
In 2013, before the second SPTpol observing season commenced, we also installed larger ``side shields" that reach from either side of the guard ring to the front edge of the telescope.

The SPTpol focal plane is composed of 1536 feedhorn-coupled transition edge sensor (TES) detectors:
360 detectors in 180 polarization-sensitive pixels at 95\,GHz, and 1176 detectors in 588 polarization-sensitive pixels at 150\,GHz.
More details about the design and fabrication of the 95 and 150\,GHz pixels can be found in \cite{sayre12} and \cite{henning12}, respectively.
The detectors are operated in their superconducting transitions at $\sim\,500$\,mK and we use superconducting quantum interference device (SQUID) amplifiers and a digital frequency-domain multiplexing readout system \citep{dobbs12b, dehaan12} to record detector time-ordered data.
In this analysis, we use data from the 150\,GHz detectors.
We include data from 95\,GHz only when defining point sources to mask during data processing.
The full 5 yr, two-frequency SPTpol dataset will be used in future work.

\section{Observations and Data Reduction}
In this section, we describe the observations of the SPTpol $500$ survey field.
We follow with a description of the data processing pipeline that starts with detector time-ordered data and ends with a set of 125 maps we use to estimate the CMB temperature and polarization power spectra.

\subsection{Observations}
The SPTpol survey field is a $500$, patch of sky spanning 4 hr of right ascension, from 22 hr to 2 hr, and 15 degrees of declination, from $-65$
The field also overlaps the survey region of the BICEP/Keck series of experiments 
We include measurements from three seasons of dedicated CMB observations during which the Sun was below the horizon or far from our observing field: 2013 April 30 --- 2013 November 27, 2014 March 25 --- 2014 December 12, and 2015 March 27 --- 2015 October 26.
Over 9087 hr of dedicated observations, the field was independently mapped 3491 times.
A fourth season of observations on this field ended in 2016 September, but these data are currently under study and are not included in this analysis.

A single observation of the field consists of either 106 or 109 constant-elevation raster scans depending on the observing strategy discussed below, with the telescope first scanning right and then left.
After each right/left scan pair, the telescope makes a step in elevation of either $9.2$ or $9.0e$ before making another set of paired scans.
This process repeats until the field is completely mapped once, and we define the corresponding set of scans as a single ``observation." 

Over the observation period covered here, we used two strategies to observe the field, ``lead-trail" and ``full-field" observing.  
From the beginning of observations through 2014 May 29, we mapped the field using an azimuthal lead-trail strategy, similar to that described in C15.
In this observing mode, the field is split into two equal halves in right ascension, a ``lead'' half-field and a ``trail'' half-field.
The lead half-field is observed first over a period of 2 hr, followed immediately by a 2 hr trail half-field observation, with the scan speed and elevation steps defined such that the lead and trail observations occur over the same azimuth range.  
For lead-trail observations, we scan the half-fields at a rate of 1.09 degrees per second in azimuth, or 0.59 degrees per second on the sky at the central declination of the field.  
To increase sensitivity to larger scales on the sky, on 2014 May 29 we switched to mapping the field with a full-field strategy, where constant-elevation scans are made across the entire range of right ascension of the field over a 2 hr period.
To reduce noise at low multipoles, corresponding to larger scales on the sky, we increased the scanning speed to 2 degrees per second in azimuth, or 1.1 degrees per second on the sky.
Higher scanning speeds move sky signals of interest to higher temporal frequencies, away from instrumental $1/f$ noise.

In addition to CMB field observations, we also routinely take a series of measurements for calibration and data quality control.  See \citet{schaffer11} and C15 for more details.

\subsection{Time stream Processing}
The raw data are composed of digitized, time-ordered  data, or ``time streams," for each detector in the focal plane.
These time streams are filtered before making maps to remove low-frequency signal from the atmosphere, instrumental $1/f$ noise, and scan-synchronous structure, as well as to reduce high-frequency signals that could alias down into the signal band when binning.

We Fourier-transform the time streams to apply a low-pass filter and to downsample the data by a factor of 2 to reduce computational requirements.
Three harmonics of two spectral lines originating from the pulse tube coolers, needed to cool the instrument to cryogenic temperatures, are notch-filtered at this time.
We calculate the filtered detector power spectral densities (PSDs) to determine inverse-noise-variance weights for mapmaking.

On a scan-by-scan basis, we subtract Legendre polynomial modes from each detector's time stream.
For lead-trail observations, we perform a fifth-order polynomial subtraction, while we use a ninth-order polynomial subtraction on full-field observations since the observations are twice as long in right ascension.
This filtering step performs an effective high-pass filter on the data at $ 50$ in the scan direction, which sets the lower multipole bound for this analysis.
(Since the telescope is located at one of the geographic poles and each scan is performed at a constant elevation, time stream filtering only removes modes in the direction of the scan on the sky, from right ascension.)
Additionally, if during a scan a detector passes within $5$ of an extragalactic source with unpolarized flux $> 50$\,mJy at either 95 or 15 the relevant time stream samples are masked during polynomial filtering.
To remove power from higher multipoles that would alias into the signal band through map pixelization, we also perform a temporal frequency low pass on the time streams that corresponds to n the telescope scan direction.


\section{Dark Energy Survey}

\end{document}