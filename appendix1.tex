\chapter{Appendix A}
\section{Section in an appendix}

\section{Simulated skies}
\label{sec_appendix_simulated_skies}
The MLE presented in this work depends being able to produce large numbers of realistic simulated skies, incorporating a diverse range of as\
trophysical signals: the CMB (lensed by the galaxy cluster), radio galaxies, dusty galaxies, the kSZ effect, and the tSZ effect.
One unique challenge for cluster studies is that the galaxy cluster itself can source most of these signals in addition to contributions fro\
m other unrelated haloes. These simulated skies are used for the calculation of the pixel-pixel covariance matrix, and for the creation of m\
ock data sets (\S\ref{sec_pixel_pixel_cov_matrix}). In this appendix, we detail the creation of these these sky simulations.

The sequence of operations is as follows. First, simulations of each signal are created on $50^\prime \times 50^\prime$ boxes with a $0.25^\prime$ pixel resolution. Most of this appendix will focus on how this is done. The CMB maps are then lensed by the galaxy cluster, convolved\
 by a Gaussian beam of the appropriate size, and rebinned to $0.5^\prime$ pixels. This final rebinning reduces the number of map pixels four\
-fold and substantially speeds up the MLE without significantly reducing the SNR. White, Gaussian instrumental noise is added, with a pixel \
RMS level based on an experiment's sensitivity, observing time and survey area. For computational reasons, the $50^\prime \times 50^\prime$ \
box is cut down to the central $10^\prime \times 10^\prime$ area used in the analysis. Finally we point out a subtle effect because of const\
raining the simulations to a $50^\prime \times 50^\prime$ box. Choosing a smaller box will reduce the background CMB gradient and subsequent\
ly the lensing signal generated by the cluster -- which will tend to worsen $\Delta M/M$ --. To quantify this, we repeated our simulations w\
ith a larger $2^{\circ} \times 2^{\circ}$ such that it encompasses the first peak of the CMB. At $\Delta T = 1 \mu K-arcmin$ for a sample of\
 100,000 clusters, the mass uncertainty $\Delta M/M$ is now 0.237\% as opposed to 0.252\% in the left panel of Fig \ref{fig_delM_M_1000_clusters_T_QU_EB_ideal_FG}, a very small effect.

Since we are dealing with very small areas of sky, we adopt the flat-sky approximation and substitute Fourier transforms for spherical harmo\
nic transforms. The Fourier wavenumber $k$ is related to the multipole $\ell$ by $k = \sqrt{k_x^{2} + k_y^{2}} = \frac{\ell}{2\pi}$. We defi\
ne the azimuthal angle $\phi_{\ell}$ as $tan^{-1}(k_{y} / k_x)$.\subsection{Cosmic Microwave Background}
\label{sec_appendix_CMB}

To simulate CMB maps that have been lensed by a massive galaxy cluster, we begin by creating T, Q, and U maps that are Gaussian realizations \citep{kamionkowski1996} of the CMB power spectra ($C_{\ell}^{TT}, C_{\ell}^{TE}, C_{\ell}^{EE},$ and $C_{\ell}^{BB}$). 
\iffalse{Note that the Q and U spectra are set by $C_{\ell}^{EE}$ and $C_{\ell}^{BB}$ according to:
\begin{eqnarray}
Q_{\ell} &=& E_{\ell}\ cos (2 \phi_{\ell}) - B_{\ell}\ sin (2 \phi_{\ell})\\
U_{\ell} &=& E_{\ell}\ sin (2 \phi_{\ell}) + B_{\ell}\ cos (2 \phi_{\ell})
\label{eq_QU_from_EB}
\end{eqnarray}}\fi
For these fiducial power spectra, we use the lensed CMB power spectra predicted by \texttt{CAMB} \citep{lewis2000} for the \emph{Planck} 2015 $\Lambda$CDM cosmology\footnote{More specifically, we use the best-fit parameters from the \emph{Planck} 2015 chain that combines the \emph{Planck} 2015 temperature, polarization, lensing power spectra with BAO, $H_{0}$, and SNe data (\texttt{TT,TE,EE+lowP+lensing+ext} in Table 4 of \cite{planckcosmo2015}).}. Note that the tensor-to-scalar ratio $r$ is zero in this chain, and there is no contribution from inflationary B-modes \citep{baumann2009}. By using Gaussian realizations of the lensed CMB power spectra, we are effectively assuming (1) that the lensing due to large-scale structures (LSS) occurs at higher redshift than the galaxy cluster, and (2) that the small non-Gaussianities due to this LSS lensing are negligible. We then lens the T, Q, and U maps using with the cluster convergence profile described in \S\ref{sec_cluster_profile}. 
We deal with sub-pixel deflection angles by interpolating over the maps using a fifth-degree B-spline interpolation. 

\subsection{Sunyaev-Zel'dovich (SZ) effect}
\label{sec_appendix_SZ}

There are two SZ effects of interest: the kinematic SZ (kSZ) effect and the thermal SZ (tSZ) effect. Both SZ effects have contributions from the cluster itself as well as from unrelated haloes. For the latter signal, we assume the best-fit tSZ and kSZ power spectra from \citet{george2015}. 
We use the \citet{shaw2010} model for the tSZ power spectrum with an overall normalization of \mbox{$D_{\ell=3000}^{^{\rm tSZ, 150 GHz}} = 3.7\ \mu K^{2}$}. For the kSZ spectrum, we take the \citet{shaw2012} model  with an overall normalization of \mbox{$D_{\ell=3000}^{^{\rm kSZ}} = 2.9\, \mu K^{2}$}. For the uncorrelated tSZ and kSZ signals, we simply create Gaussian realizations of these two spectra. 

In some cases, we wish to study the effect of the kSZ and tSZ signals from the galaxy cluster in question. %At these times, we replace the simple Gaussian realizations by cutouts around comparable mass haloes in simulated SZ maps. 
At these times, we add the cutouts around comparable mass haloes in simulated SZ maps to the simple Gaussian CMB realizations. We use the kSZ maps provided for the \citet{flender2016} N-body simulations, while for tSZ maps, we use the smoothed-particle hydrodynamics (SPH) simulations of  \citet{mccarthy2013}. 

We ignore the extremely small polarized SZ signals in all cases. To generate polarization the free electrons of the intracluster medium must be exposed to a quadrupole radiation field, due to the CMB quadrupole mode for tSZ polarization, $p_{tSZ}$, and to an apparent CMB quadrupole created by the Doppler effect of bulk velocities in the electrons for kSZ polarization, $p_{kSZ}$. The level of the tSZ polarization is $p_{tSZ} \sim 0.1 (\tau_{e}/0.02)\ \mu K$ while the kSZ polarization level is $p_{kSZ} \sim 0.1 \beta_{t}^{2}\tau_{e}\ K$ \citep{sazonov1999, carlstrom2002} where $\tau_{e}$ is the optical depth of the cluster and $\beta_{t} = v/c$ transverse component of the electron's velocity. The clusters used in this work have $M_{200} =  2 \times 10^{14}\ M_{\odot}$ and an expected optical depth of $\tau_{e} \sim 0.004$ \citep{flender2016b}. 
We assume a velocity, $v = 1000\ km\ s^{-1}$, leading to $p_{tSZ}=20$\,nK and $p_{kSZ} =2$\,nK. This level of polarization is negligible.

\subsection{Radio and Dusty Galaxies}
\label{sec_appendix_extragal}

We also create simulated maps of radio and dusty galaxies. 
These maps include four terms: 
\begin{enumerate}
\item[1.] radio galaxies following a spatial Poisson distribution ($C_{\ell}^{^{radio}} \propto {\rm constant}$),
\item[2.] dusty star forming galaxies (DGs) also following a spatial Poisson distribution ($C_{\ell}^{^{\rm DG-Po}} \propto {\rm constant}$),
\item[3.] clustering of DGs ($C_{\ell}^{^{\rm DG-clus}} \propto  \ell^{0.8}$), and 
\item[4.] the overdensity of DGs in galaxy clusters
\end{enumerate} 
The first three items are uncorrelated with the galaxy cluster and are expected to reduce the SNR of the estimators without biasing the reconstructed lensing mass. However, the relative overdensity of DGs in galaxy clusters can potentially bias the derived masses, and is discussed  in \S\ref{sec_sys_bias_checks}. We assume the galaxies are randomly polarized at the 4\% level, based on \citet{sptpol_delensing_2017}. This 4\% level is expected to be an over-estimate for the DGs \citep{seiffert2007}. 

For radio galaxies, we draw a Poisson realization of the number counts as a function of flux \citep{dezotti2005}. We ignore the clustering of radio galaxies as it is irrelevant at frequencies observed by the CMB telescopes \citep{dezotti2005, george2015}. We take a more sophisticated approach for the dusty galaxies to handle the clustering and tSZ-CIB correlation. We begin by taking a Poisson distribution for the unusually bright dusty galaxies ($>1$\,mJy) \citep{bethermin2011} . For fainter DGs, we create a set of number density contrast maps $\delta(\hat{\textbf{n}})^{^{\rm DG-Po}}$ covering narrow flux bins. We adjust these number density contrast maps to account for the desired clustering and correlation properties as outlined below, and then assign random fluxes to each pixel in Jy/sr (drawn uniformly across the flux bin). The resulting flux maps are then converted to CMB temperature units ($\ \mu K_{\rm CMB}$). 

For the tSZ-CIB correlation, we use the tSZ simulations produced by \cite{mccarthy2013} and the $C_{\ell}^{^{\rm tSZ \times CIB}}$ cross spectrum measured by \citet{george2015}. Using these we produce a $\rm tSZ \times CIB$ correlated map $T(\hat{\textbf{n}})^{^{\rm tSZ \times CIB}}$ and a weight map $W(\hat{\textbf{n}})^{^{\rm DG \times tSZ}}$ to modify $\delta(\hat{\textbf{n}})^{^{\rm DG-Po}}$ as
\begin{eqnarray}
T_{\ell} \equiv T_{\ell}^{^{\rm tSZ \times CIB}} & = & T_{\ell}^{^{\rm tSZ}}\ \frac{C_{\ell}^{^{\rm tSZ \times CIB}}}{C_{\ell}^{^{\rm tSZ}}}\\
\label{eq_DG_tSZ_weight_map}
W(\hat{\textbf{n}}) \equiv W(\hat{\textbf{n}})^{^{\rm DG \times tSZ}} & = & \frac{\left| T(\hat{\textbf{n}}) \right|}{\sum{\left|T(\hat{\textbf{n}})\right|}}\\
\label{eq_DG_tSZ_density_map}
\widetilde{\delta}(\hat{\textbf{n}})^{^{\rm DG \times tSZ}} & = & W(\hat{\textbf{n}})\ \delta(\hat{\textbf{n}})^{^{\rm DG-Po}}
\end{eqnarray} where subscripts $\ell$ refer to the harmonic space transforms of the CMB map $T(\hat{\textbf{n}})$. Eq.(\ref{eq_DG_tSZ_weight_map}) and Eq.(\ref{eq_DG_tSZ_density_map}) ensure that the number of point sources are conserved and are predominantly clustered near massive dark matter haloes. For the clustering component of DG we modify Eq.(\ref{eq_DG_tSZ_density_map}) following \cite{gonzalez2004}
\begin{eqnarray}
\widetilde{\delta}_{\ell}^{^{DG}} \equiv \widetilde{\delta}_{\ell}^{^{\rm DG-Po, clus, tSZ}} & = & \widetilde{\delta}_{\ell}^{^{\rm DG \times tSZ}}\ \frac{ \sqrt{ C_{\ell}^{^{Po}} + C_{\ell}^{^{clus}}} } { \sqrt{ C_{\ell}^{^{Po}} } }
\end{eqnarray}