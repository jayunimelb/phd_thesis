\chapter{Maximum Likelihood Estimator}
\label{ch:MLE}
\section{Overview}
Cosmic Microwave Background photons are lensed by the intervening galaxy cluster. 
On the galaxy cluster length scales (order of arcminutes) CMB can be approximated as a gradient due to diffusion damping. 
Gravitational lensing induces a dipole kind of structure on top of the gradient with hot and clod spots swapped. 
The strength of the gravitational signal (lensing dipole) is directly proportional to the background gradient and the cluster mass. 
As polarisation signal is an order of magnitude smaller than the lensing signal which means the CMB-cluster lensing signal is also an order of magnitude smaller in polarisation. 
Fig 1. shows the lensing signal for a massive cluster of mass $5*10^{14}$\msolar; left panels is the background CMB gradient for the temperature and polarisation and on the right panel we have the corresponding lensing dipole signal.

Lensing remaps the unlensed CMB temperature and polarisation fields based on the gravitational deflection angle of the cluster. 
\begin{eqnarray}
T(\hat{n}) = \tilde{T}(\hat{n} + \alpha(\hat{n}))\\
Q(\hat{n}) = \tilde{Q}(\hat{n} + \alpha(\hat{n}))\\
U(\hat{n}) =  \tilde{U}(\hat{n} + \alpha(\hat{n}))
\end{eqnarray}
where T represents the temperature field, Q and U are the stokes polarisation parameters respectively. 
Tilde represents the unlensed fields, $\alpha(\hat{n})$ is the deflection angle vector due to the cluster lensing along the $\hat{n}$ direction. 
Deflection angle is related to gravitational potential as $\nabla \phi (\hat{n})$; is directly proportional to the mass of the cluster.

In literature, there are several methods to extract lensing singal from CMB data. 
During my thesis, I mainly worked on two methods: Maximum Likelihood Estimator and Quadratic Estimator. 
In this chapter, I will explain the Maximum Likelihood estimator in detail. 
\section{Maximum Likelihood Estimator}
Gravitational lensing by a galaxy cluster induces extra pixel-pixel correlations. 
Maximum Likelihood Estimator (MLE) extracts the lensing signal by modelling these pixel-pixel correlations in the form of covariance matrix.


\subsection{Covariance matrix calculation}
As the lensing is a small effect we take into account only \smallboxsize to calculate the covariance matrix.
 We calculate the covariance matrix by using a set of simulated skies. 
 To calculate the simulated lensed CMB sky, first we generate the lensed CMB power spectra ($C^{TT}_{l}, C^{TE}_{l}, C^{EE}_{l},$ and $C^{BB}_{l}$) form CAMB for the $planck$ 2015 cosmology.  
 We generate Gaussian random realisations with these power spectra on a 50\am X 50 \am boxsize. 
 Q and U maps are generated by using E and B maps as
 \begin{equation}
U  = B
 \end{equation}
 While we only use  \smallboxsize  for final calculations we simulate a bigger box to take into account the large scale gradient.
 These Gaussian realizations are then lensed by an assumed galaxy cluster density profile (explained in next section). 
 
 With simulated lensed CMB maps in hand we calculate the covariance matrix as follows
 \begin{equation}
 \Sigma_{lens}(M,z) = 
 \end{equation}
 where vector $G_{i}$ is either the polarisation or temperature data for $i^{th}$ sky realisation. 
 The number simulated skies depend on the number of degrees of freedom in the covariance matrix. 
 In our case, we concatenate Q and U maps for our polarisation estimator for which the covariance matrix is an 800 X 800. 
 Number of simulations scale twice the number of elements in the covariance matrix; we found 1,30,000 simulations are sufficient for recovering cluster masses without any bias. 
  We then apply Hartlap correction term $\frac{(n_{sims} -n_{d} -1)}{n_{sims}}$, where $n_{sims}$ is 1,30,000 and $n_{d}$ is the length of the vector 400(800) for T(QU), to remove any possible bias in $\Sigma^{-1}_{lens}$ due to the limited number of simulations. 
  
 These simulated sky maps are also used for the estimation of systematic biases. 
 There are several astrophysical sources which act as a systematic bias and foregrounds for the CMB-cluster lensing analysis. 
 In this work we consider clusters own SZ effects such as thermal Sunayev-Zel'dovich (tSZ) and kinematic Sunayev-Zel'dovich effects. 
 tSZ is in explained in detail in the next chapter. 
 Along with these SZ effects we have also considered sources which are uncorrelated with cluster such as tSZ effect from other halos, dusty star forming galaxies (DSFGs), and radio galaxies.
 In appendix, we provide in more details about the addition of these foregrounds to simulated skies.
 
 In this chapter we have considered only simulated data to check the efficiency of MLE. 
 Unless otherwise mentioned all the clusters are simulated at a mass of $M_{200} = 2*10^{14}$\msolar \pending{footnote} and at redshift of 0.7.
 For covariance matrix calculation we simulate the skies at redshift of $0.7$ with mass resolution of $2*10^{12}$\msolar. 
 Note that such fine gridding might not be computationally feasible for data where the clusters span wide range of masses and redshifts.
 An optimal solution would be generate the covariance matrices on coarser grid of mass and redshift; then interpolating on a finer grid.
 
 
 
  
   
  
  \subsection{Likelihood Estimation}
  With the covariance matrix in hand we calculate the likelihood given the data as follows 
  
  \begin{equation}
  -2lnL(d|\Sigma_{lens}) = ln |\Sigma_{lens}| + d^{T} \Sigma^{-1}_{lens} d
  \end{equation}
  where the data vector d is the pixel values of the observed T or Q/U maps.
  The pixel values are defined as the variations from the mean CMB temperature (polarisation) and hence have zero mean.
  
  Since the majority of the lensing singal is within few arcminutes from the cluster center, we carry out the lensing analysis within \smallboxsize of the cluster center to simplify and speed up the analysis. 
  We also checked that by increasing the boxsize to 14\am we gain an improvement in a SNR of less than 1\%, however, that the increases the computational complexities. 
  As mentioned before, the lensing signal of a single galaxy cluster is much weaker to be detected. 
  We stack many clusters to increase SNR (signal to noise ratio) to a reasonable level
  \begin{equation}
  -2ln L(d| \Sigma_{lens})_{tot} = \Sigma^{n}_{i =0} w_{i} [ln |\Sigma_{lens}| + d^{T}_{i} \Sigma^{-1}_{lens}  d]
  \end{equation}
  where n is the total number of clusters in the sample, $w_{i}$ is the weight for the $i^{th}$ cluster which depends on the survey.

  
 